\documentclass{article}
\usepackage{amsmath}
\usepackage{amsfonts}
\usepackage{graphicx}
\usepackage{xcolor}

\title{Hola, mundo}
\author{Jefferson Tavara}
\date{\today}

\begin{document}

\maketitle

\section{Comenzando}

\textbf{Hola, mundo.} Hoy estoy aprendiendo \LaTeX. \LaTeX{} es un excelente lenguaje para producir documentos académicos. Puedo escribir matemáticas en línea, como \(a^2 + b^2 = c^2\). También puedo darles a las ecuaciones su propia línea:

\begin{equation}
\gamma^2 + \theta^2 = \omega^2
\end{equation}
Las ``ecuaciones de Maxwell'' son nombradas en honor a James Clark Maxwell y son las siguientes:

\begin{align}
    \vec{\nabla} \cdot \vec{E} &= \frac{\rho}{\epsilon_0} && \text{Ley de Gauss} \\
    \vec{\nabla} \cdot \vec{B} &= 0 && \text{Ley de Gauss para el magnetismo} \\
    \vec{\nabla} \times \vec{E} &= - \frac{\partial \vec{B}}{\partial t} && \text{Ley de Faraday} \\
    \vec{\nabla} \times \vec{B} &= \mu_0 \left(\epsilon_0 \frac{\partial \vec{E}}{\partial t}+\vec{J} \right) && \text{Ley de Ampere}
\end{align}
Las ecuaciones \textcolor{blue}{2, 3, 4} y  \textcolor{blue}{5} son algunas de las más importantes en Física.

\section{¿Qué hay sobre las ecuaciones matriciales?}

\[
\begin{pmatrix}
    a_{11} & a_{12} & \cdots & a_{1n} \\
    a_{21} & a_{22} & \cdots & a_{2n} \\
    \vdots & \vdots & \ddots & \vdots \\
    a_{n1} & a_{n2} & \cdots & a_{nn}
\end{pmatrix}
\begin{bmatrix}
    v_1 \\
    v_2 \\
    \vdots \\
    v_n
\end{bmatrix}
=
\begin{matrix}
    w_1 \\
    w_2 \\
    \vdots \\
    w_n
\end{matrix}
\]

\section{Tablas y Figuras}

Crear una tabla no es muy diferente de crear una matriz:

\begin{table}
\centering
\caption{Mi primera tabla.}
\begin{tabular}{|c|c|c|c|}
\hline
$x$ & 1 & 2 & 3 \\
\hline
$f(x)$ & 4 & 8 & 12 \\
\hline
f(x) & 4 & 8 & 12 \\
\hline
\end{tabular}
\end{table}

\begin{figure}
\centering
\includegraphics[width=0.7\textwidth]{Figura1.png}
\caption{Cualquier imagen.}
\end{figure}
\end{document}
